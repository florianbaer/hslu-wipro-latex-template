%-----------------------------
% Master LaTeX Dokument
%-----------------------------

%-----------------------------
% Konfiguration des Dokuments
%-----------------------------

\documentclass[a4paper,11pt]{scrreprt}

%-----------------------------
% Installierte Pakete
%-----------------------------

\usepackage[ngerman]{babel}				% Rechtsschreibekorrektur Deutsch https://ctan.org/pkg/babel

\usepackage[utf8]{inputenc}				% Unicode Support https://ctan.org/pkg/inputenc

\usepackage[a-2b,latxmp]{pdfx}			% PDFA Support https://ctan.org/pkg/pdfx

\usepackage{graphicx}					% Einfügen von Bildern https://ctan.org/pkg/graphicx?lang=en

\usepackage{tabularx}					% Erweiterte Optionen für Tabellen https://ctan.org/pkg/tabularx

\usepackage{enumitem}					% Aufzählungen https://ctan.org/pkg/enumitem

\usepackage{caption}					% Erweiterte Optionen für Captions https://ctan.org/pkg/caption

\usepackage{pdfpages}					% PDF Dokumente einbetten https://ctan.org/pkg/pdfpages

\usepackage{fancyhdr}					% Für custom Kopf- und Fusszeilen
\pagestyle{fancy}

\usepackage[acronym]{glossaries}		% Für die Erstellung eines Glossars, wir rufen glossaries ohne [toc] auf weil sonst das Abkürzungsverzeichnis im Inhaltsverzeichnis erscheint und die HSLU das nicht möchte.
\loadglsentries{Glossar.tex}			% Das ausgelagerte Glossar wird aus der Datei "Glossar.tex" geladen.

\makeglossaries

\usepackage{amssymb}

\usepackage[T1]{fontenc}				% Fontencoding https://www.ctan.org/pkg/fontenc

\usepackage[onehalfspacing]{setspace}	% 1.5-facher Zeilenabstand wie von der HSLU gewünscht

\usepackage[style=apa]{biblatex}		% Zitierung nach APA Standard
\addbibresource{Referenzen.bib}

\usepackage{hyperref}					% Formatieren von Links

%-----------------------------
% Inhalt des Dokuments
%-----------------------------

\begin{document}
	
\fancyfoot[OR]{\thepage}

\pagenumbering{gobble}
{Diese erste Seite bzw. Frontseite ist frei gestaltbar.}
\newpage

\noindent
\fontsize{12}{14}
\textbf{Wirtschaftsprojekt an der Hochschule Luzern -- Informatik} \\ \vspace*{0.6cm}

\fontsize{10.5}{12}
\noindent
\textbf{Titel:} \\ \vspace*{0.2cm}

\noindent
\textbf{Studentin/Student:} \newline \newline
\textbf{Studentin/Student:} \newline \newline
\textbf{Studiengang:} BSc Informatik oder Wirtschaftsinformatik  \newline \newline
\textbf{Jahr:} \newline \newline
\textbf{Betreuungsperson:} \newline \newline
\textbf{Expertin/Experte:} \newline \newline
\textbf{Auftraggeberin/Auftraggeber:} \newline \newline \newline
\textbf{Codierung / Klassifizierung der Arbeit:}\\
$\square$ A: Einsicht 	(Normalfall) \\
$\square$ B: R\"ucksprache	(Dauer: \ \ \ \ \        Jahr / Jahre)\\
$\square$ C: Sperre	(Dauer: \ \ \ \ \        Jahr / Jahre)\\


%%% you can use \boxtimes for filling a cross inside the square
%%% e.g., $\boxtimes$ A: Einsicht 	(Normalfall) 


\paragraph{\textbf{Eidesstattliche Erkl\"arung}}
Ich erkl\"are hiermit, dass ich/wir die vorliegende Arbeit selbst\"andig und ohne unerlaubte fremde Hilfe angefertigt haben, alle verwendeten Quellen, Literatur und andere Hilfsmittel angegeben haben, w\"ortlich oder inhaltlich entnommene Stellen als solche kenntlich gemacht haben, das Vertraulichkeitsinteresse des Auftraggebers wahren und die Urheberrechtsbestimmungen der Fachhochschule Zentralschweiz (siehe Merkblatt \flqq Studentische Arbeiten\frqq\ auf MyCampus) respektieren werden. \newline \newline
Ort / Datum, Unterschrift	\underline{\hspace*{8cm}} \newline \newline
Ort / Datum, Unterschrift	\underline{\hspace*{8cm}} \newline \newline \newline
\textbf{Ausschliesslich bei Abgabe in gedruckter Form: \\
Eingangsvisum durch das Sekretariat auszuf\"ullen}\newline \newline
Rotkreuz, den	\underline{\hspace*{4cm}}	\hspace*{1cm} Visum:	\underline{\hspace*{4cm}}

\newpage

\pagenumbering{roman}

\begin{abstract}
Hier steht ein Beispieltext. \\
Zudem wird auf den Glossareintrag zu Wipro verwiesen. \gls{wipro}

Tastaturen werden typischerweise über \gls{usb} angeschlossen. Nicht aber an der \gls{hslu}

Ah doch, jetzt sehe ich das sie an der \gls{hslu} auch \gls{usb} verwenden.

% TODO: Hier den Beispieltext löschen und eigenen Abstract schreiben
\end{abstract}

\tableofcontents

\printglossary[type=\acronymtype, title=Abkürzungsverzeichnis]

\clearpage

\pagenumbering{arabic}

\fancypagestyle{plain}{%
	\renewcommand{\headrulewidth}{0pt}%
	\fancyhf{}%
	\fancyfoot[R]{\thepage}%
}

\chapter{Problem, Fragestellung, Vision}

%TODO: Beispieltext löschen und durch eigenen Text ersetzen.

Hier wird auf das zu bearbeitende Problem eingegangen. Zudem wird auf den Glossareintrag zu Wipro verwiesen. \gls{wipro}

Tastaturen werden typischerweise über \gls{usb} angeschlossen. Nicht aber an der \gls{hslu}

Ah doch, jetzt sehe ich das sie an der \gls{hslu} auch \gls{usb} verwenden.

\chapter{Stand der Technik}

% TODO: Dieses Kapitel kann auch "Stand der Forschung" oder "Stand der Praxis" heissen. Entscheide selbst passend auf deine Arbeit.

\chapter{Ideen und Konzepte}

%TODO: Beispieltext löschen und durch eigenen Text ersetzen.

Hier geht es um die Fragestellung, wieSie die formulierten Ziele der Arbeit erreichen wollen. Sie halten z.B. erste, grobe Ideen, skizzenhafte Lösungsansätzefest. Gibt es mehrere Wege, Ansätzeum dieses Ziel zu erreichen, begründen Sie hier, warum Sie einen bestimmten Weg einschlagen. Beispiel für ein Softwareprojekt: Erste Gedanken über einegrobe Systemarchitektur. Ist z.B. eine Microservice-Architektur angebracht? Welche Alternativen bestehen, wo gibt es Problempunkte? Die Umsetzung, die Beurteilung der Machbarkeit und die detaillierte Beschreibung der umgesetzten Architektur sinddann Teil der Realisierung. Abgrenzung zu Kapitel 5:-Besteht ein wesentliches Projektziel darin, für Ihre Kunden z.B. ein Security-Konzept, ein Kommunikations-Konzeptes, ein IT-Fachkonzept oder einanderes Fach-Konzeptzuerstellen, dann wird die Entwicklungdieser(fachlichen) Konzepte unter «Realisierung» beschrieben(sie sind ja der eigentliche Kern Ihrer Arbeit).-Besteht z.B. ein wesentliches Ziel der Arbeit darin, eine passende Software-Architektur zuevaluieren, dann gehören die entsprechenden Beschreibungen ins Kapitel 5.

\chapter{Methoden}

\chapter{Realisierung}

%TODO: Beispieltext löschen und durch eigenen Text ersetzen.

Dies ist das Hauptkapitel Ihrer Arbeit!Hier wird die Umsetzung der eigenen Ideen und Konzepte(Kapitel 3)anhand der gewählten Methoden (Kapitel 4) beschrieben, inkl. der dabei aufgetretenenSchwierigkeiten und Einschränkungen.

\chapter{Evaluation und Validation}


\chapter{Ausblick}

\appendix

\addcontentsline{toc}{chapter}{Glossar} 		% Macht das "Glossar" im Inhaltsverzeichnis erscheint.
\printglossary[style=altlist,title=Glossar]

\listoffigures

\listoftables

\printbibliography

\end{document}
